%\documentclass[11pt,a4paper]{report}

\usepackage[margin=1in]{geometry}
\usepackage{polski}
\usepackage{ragged2e}
\usepackage[utf8]{inputenc} 
\usepackage{graphicx}



\title{\LARGE{Politechnika Śląska\\
		Wydział Automatyki, Elektroniki i Informatyki\\}
		\LARGE{\textbf{Podstawy Programowania Komputerów}\\}
		\textbf{Czerwono–czarni}
}
\date{}

\input{refman.tex}
\begin{document}
%\maketitle
	\begin{center}

	
\rmfamily
	\end{center}

		autor Aleksander Augustyniak					\\
		prowadzący mgr inż. Marek Kokot				\\
		rok akademicki 2019/2020						\\
		kierunek informatyka								\\
		rodzaj studiów SSI									\\
		semestr 1											\\
		termin laboratorium piątek, 11:45 – 13:15		\\
		sekcja 22												\\
		termin oddania sprawozdania 2020-01-27		\\

%blank page

\section{Treść zadania}
\begin{justify}
Napisać program sortujący liczby rzeczywiste w pewnym zbiorze. Liczby podawane są w dość specyficzny sposób. Liczba może być dodana lub usunięta ze zbioru. Dodanie liczb jest realizowane przez komendę \texttt{add}, po której może wystąpić jedna lub więcej liczb (rozdzielonych białymi znakami). Komenda \texttt{remove} usuwa podaną po niej liczbę (lub liczby rozdzielone białymi znakami) ze zbioru. Komenda \texttt{print} powoduje wypisane liczb zawartych w zbiorze w porządku rosnącym. Po komendzie tej można podać nazwę pliku, wtedy wartości zostaną zapisane do tegoż pliku zamiast na standardowe wyjście. Komenda \texttt{graph} wypisuje drzewo w postaci graficznej – głębsze poziomy drzewa są wypisywane z coraz większym wcięciem. Dodatkowo wartości w węzłach czarnych są wypisywane w nawiasach kwadratowych, np. \texttt{[13]}, w węzłach czerwonych – w nawiasach okrągłych, np. \texttt{(13)}. Podobnie jak w przypadku komendy print po komendzie graph można podać nazwę pliku do zapisu. Jeżeli komendy \texttt{print} i \texttt{graph} są użyte do zapisu do pliku, możliwe jest poprzedzenie nazwy pliku znakiem \texttt{+}. Wtedy plik nie zostanie nadpisany, ale nowa treść zostanie dopisana na końcu pliku, nie niszcząc dotychczasowej jego zawartości. Znak \texttt{\%} rozpoczyna komentarz do końca linii. Każda komenda jest zapisana w osobnej linii. Jeżeli zostanie podana niepoprawna komenda, program ignoruje ją. \par

Przykładowy plik wejściowy:\\
\texttt{
\% przykladowy plik wejsciowy\\
add -3.14 43.9 4\\
graph \% wypisane z wcieciami i z oznaczeniami kolorow wezlow \\
remove 4\\
print \% wypisane na ekran\\
add 3.45 -0.32\\
print test-1.txt \% wypisanie do pliku\\
\\
add 490 32.3\\
\\
print \% na ekran \\
print + test-1.txt \% dopisanie do pliku\\
\\
\textrm{Po komendzie \texttt{graph} zostanie wypisane drzewo:} \\
\indent{(-3.14)}\\
{[4]}\\
\indent(42.9) \\
\break
}
Program uruchamiany jest z linii poleceń z wykorzystaniem jednego przełącznika: \par
\texttt {-i plik wejściowy} \\
Do przechowywania liczb należy wykorzystać drzewa czerwono-czarne. Jest to warunek sine qua non.\\
\end{justify}

\section{Analiza zadania}
\begin{justify}	
Zadanie skupia się wokół problemu sortowania liczb rzeczywistych, zapisanych w pliku o podanej przez użytkownika nazwie.
	
\end{justify}

\subsection{Struktury danych}
\begin{justify}
Program wykorzystuje drzewo czerwono–czarne do przechowywania wartości. Wartość jest przypisywana węzłowi o odpowiednim kolorze.

Podstawowe zasady panujące w drzewie czerwono–czarnym: \cite{zasady_drzewa}
\begin{enumerate}
	\item{Każdy z węzłów jest czerwony lub czarny.}
	\item{Korzeń drzewa jest czarny.}
	\item{Wszystkie liście (NIL) są czarne.}
	\item{Jeżeli węzeł jest czerwony, to oba jego potomki są czarne.}
	\item{Każda ścieżka od wybranego węzła do liścia zawiera tę samą ilość czarnych węzłów.}	
\end{enumerate}

Powyższe zasady – opisujące wzajemny stosunek węzłów drzewa – pomagają zapanować nad wartościami przechowywanymi w węzłach oraz równoważyć jego wysokość. Pozwala to na skrócenie czasu poszukiwania, usuwania oraz dodawania wartości do struktury.

%\begin{figure}
%\centering
%\includegraphics[width=7cm]K.
%\caption{Wizualizacja drzewa czerwono–czarnego}
%\label{fig:obrazek K}
%\end{figure}


\end{justify}

\subsection{Algorytmy}
\begin{justify}

Dodawanie wartości do drzewa czerwono–czarnego jest realizowane podobnie, jak w drzewie poszukiwań binarnych, lecz dodatkowo program wykonuje lokalne operacje na węzłach, takie jak: rotacje w lewo/prawo oraz zmiana kolorów węzłów, by przywrócić właściwości drzewa czerwono–czarnego.

Gwarantowana złożoność obliczeniowa wynosi średnio, jak i w najgorszym przypadku $O(\log{\emph{n})}$, gdzie \emph{n} oznacza ilość wszystkich elementów znajdujących się w strukturze danych.

By zrealizować operacje wypisywania i usuwania wartości węzłów, program rekurencyjnie przechodzi przez wszystkie węzły drzewa.

\end{justify}

\section{Specyfikacja zewnętrzna}
\begin{justify}

Aby uruchomić program, należy wprowadzić do linii poleceń, w odpowiedniej kolejności: \\
\indent\texttt{nazwa\_programu -i plik\_wejściowy} \\
Jeżeli program nie został prawidłowo uruchomiony, pojawi się stosowny komunikat z ewentualną instrukcją prawidłowego uruchomienia programu.





\end{justify}

\section{Specyfikacja wewnętrzna}
\begin{justify}

\end{justify}

\subsection{Ogólna struktura programu}
\begin{justify}
Funkcja główna sprawdza, czy ilość wprowadzonych parametrów do linii poleceń jest większa od dwóch. Jeżeli nie – program kończy się – w przeciwnym wypadku zostaje uruchomiona funkcja \textbf{ReadFromFile}, przyjmująca jeden plik wejściowy. Funkcja ta zczytuje linia po linii każde polecenie i wywołuje odpowiednie podfunkcje: Add, Delete, Print, Graph, itd. – w zależności od znajdującej się w pliku wejściowym komendy.

Program – po wczytaniu wszystkich komend – dealokuje całą zajętą przez drzewo pamięć i zamyka się.

\end{justify}

\subsection{Szczegółowy opis typów i funkcji}
\begin{justify}
Szczegółowy opis typów i funkcji znajduje się w załączniku.
\end{justify}

\section{Testowanie}
\begin{justify}


Program został przetestowany pod względem wycieków pamięci.
\end{justify}

\section{Wnioski}
\begin{justify}

\end{justify}

\begin{thebibliography}{9}
\bibitem{zasady_drzewa} 
Cormen, Thomas; Leiserson, Charles; Rivest, Ronald; Stein, Clifford (2009). "13". Introduction to Algorithms (3rd ed.). MIT Press. pp. 308–309. ISBN 978-0-262-03384-8.

\bibitem{}

\end{thebibliography}




\end{document}
